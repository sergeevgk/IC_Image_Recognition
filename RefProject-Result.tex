\section{Результаты}

\imgh{{100pt}}{original-1.png}{Снимок с ИК-камеры}

\imgh{{100pt}}{Gray-1.png}{Снимок с ИК-камеры в оттенках серого цвета}


\subsection{Операторы Лапласа и Собеля}

\imgh{{100pt}}{Laplace-diff-1.JPG}{Результат применения оператора Лапласа}

\imgh{{100pt}}{Sobelx-diff-1.JPG}{Результат применения оператора Собеля по горизонтальной оси}

\imgh{{100pt}}{Sobely-diff-1.JPG}{Результат применения оператора Собеля по вертикальной оси}

Полученные в результате применения операторов изображения обладают достаточно низким качеством. На черно-белых изображениях, полученных функцией $convertScaleAbs$, можно визуально определить границы структурных элементов схемы, но сделать это довольно проблематично. Методы детектирования прямых, окружностей и контуров в данном случае не дают значимых результатов.


\subsection{Оператор Кэнни}

\imgh{{100pt}}{Canny-RGB-1.png}{Результат применения оператора Кэнни, нулевой пороговый уровень}

\imgh{{100pt}}{Canny-8bit-1.png}{Приведение результата к 8-битному черно-белому изображению}

Как видно из полученных изображений, оператор Кэнни дает гораздо более точный результат детектирования границ. На цветном изображении видны переходы от цвета к цвету, и соответственно от одной температуры к другой, которые не сливаются в одну границу. Черно-белое изображение менее информативно для человека, но необходимо для детектирования объектов методами OpenCV. 
\newline 
Оператор Кэнни позволяет ввести 2 пороговых уровня, с помощью которых отбирает пиксели, которые в итоге составят границы на изображении. Рассмотрим 2 порога $low\_threshold$: 22 и 76.

\imgh{{100pt}}{Canny-RGB-1-1.png}{Результат применения оператора Кэнни, пороговый уровень 22}

\imgh{{100pt}}{Canny-8bit-1-1.png}{Приведение результата к 8-битному черно-белому изображению}

\imgh{{100pt}}{Canny-RGB-1-2.png}{Результат применения оператора Кэнни, пороговый уровень 76}

\imgh{{100pt}}{Canny-8bit-1-2.png}{Приведение результата к 8-битному черно-белому изображению}

Видно, что при введении порогового уровня снижается количество линий на изображении. С одной стороны, это позволяет убрать лишние границы, но с другой -- могут потеряться важные пиксели, которые составляют искомые границы объектов на изображении. Поэтому необходимо придумать иной способ пороговой фильтрации.
\newline

\subsection{Пороговая фильтрация по цвету}
Обратимся снова к RGB-результату действия оператора Кэнни. Видно, что цвета на изображении достаточно точно выделяют границы составляющих частей платы. 
\newline 
Попробуем разделить изображение на несколько цветовых диапазонов. Так как снимок получен с тепловизора, то цветовой диапазон однозначно соответствует температурному диапазону, а значит такой фильтр выделяет изотермы -- области изображения с одинаковой температурой. Благодаря такому подходу удастся выделить как самые холодные части (крепления, части без проводки и нагревающихся элементов), так и самые горячие -- процессоры, вычислительные схемы и элементы питания. 
\newline
Изображение было разбито на 6 диапазонов (синий, зеленый, желтый, оранжево-красный, розовый и белый). Ниже представлены результаты действия фильтра и попытки выделения прямых и окружностей на отфильтрованном по температурному диапазону изображении. Диапазоны указаны тройками в формате HSV, поскольку выделение RGB-цветов в этом представлении более удобное и функции преобразования RGB<=>HSV присутствуют в библиотеке OpenCV. (H - hue, S - saturation, V - value)
\newline

\imgh{{100pt}}{detected-circles-blue.png}{Синий диапазон: [(97, 250, 0), (115, 255, 255)], круги}

\imgh{{100pt}}{detected-lines-blue.png}{Синий диапазон, линии (выделение зеленым)}

\imgh{{100pt}}{detected-circles-green.png}{Зеленый диапазон: [(25, 70, 118), (86, 255, 255)], круги}

\imgh{{100pt}}{detected-lines-green.png}{Зеленый диапазон, линии (выделение красным)}

\imgh{{100pt}}{detected-lines-yellow.png}{Желтый диапазон: [(16, 130, 88), (39, 255, 255)], линии (выделение красным)}

\imgh{{100pt}}{detected-lines-red.png}{Оранжево-красный диапазон: [(0, 130, 195), (16, 255, 255)], линии (выделение зеленым)}

\imgh{{100pt}}{detected-circles-pink-1.png}{Розовый диапазон: [(0, 74, 200), (17, 206, 255)], круги}

\imgh{{100pt}}{detected-circles-pink-2.png}{Розовый диапазон, круги(2)}

\imgh{{100pt}}{detected-lines-pink.png}{Розовый диапазон, линии (выделение зеленым)}

\imgh{{100pt}}{detected-circles-white-1.png}{Белый диапазон: [(0, 1, 251), (255, 69, 255)], круги}

\imgh{{100pt}}{detected-circles-white-2.png}{Белый диапазон: [(0, 2, 250), (111, 2, 255)], круги. Высокая фильтрация, оставляет только один контур}

Представленные результаты получены на изображениях, где обнаружение кругов или линий приносило полезный эффект, позволяющий выделить "значимые" $\;$области схемы.
\newline
Детектирование проводилось методом детектирования Хафа (методы $HoughCircles$ и $HoughLines$ библиотеки OpenCV). Также предпринимались попытки обнаружить контуры методом $findContours$ той же библиотеки, однако к значимым результатам они не привели.
\newline
Подбор параметров проводился вручную, исходя из их смысла и текущих результатов. Параметры для детектирования кругов и линий сохранены в отдельные структуры для воспроизводимости результата.

\subsection{Попытка улучшения результата}
Как можно увидеть на изображении "Оранжево-красный диапазон"$,$ небольшая микросхема с большой температурой, расположенная посередине платы, не была обнаружена. Один из способов борьбы с такими ситуациями -- сужение цветового интервала. Однако, если слишком сильно сузить интервал, то замкнутые контуры могут потерять слишком большое число точек и перестанут обнаружиться вовсе.
\newline 
В случае с небольшой микросхемой данный способ сработал. На иллюстрации ниже можно увидеть, что сужение интервала без изменения каких-либо параметров алгоритма Хафа оказалось достаточным для обнаружения микросхемы.

\imgh{{100pt}}{detected-circles-red.png}{Красный диапазон: [(0, 130, 195), (13, 255, 255)], круги}

\subsection{Обсуждение результата}
По итогам проведенного исследования получен набор изображений с отмеченными границами элементов. Большая часть желаемых контуров при помощи исследуемых методов не была выделена, однако некоторые главные части изображения всё же были приблизительно отмечены. 
\newline
Несомненно, для оптимального и точного детектирования элементов технического изображения необходимо использовать нейросеть. Ручной перебор параметров не даст необходимой точности и займет очень много времени. Применение нейросети возможно как на этапе выбора цветовых фильтров, так и на этапе выбора параметров для методов обнаружения контуров, линий, кругов и других геометрических примитивов.
\newline
Следующим этапом исследования будет поиск готовых решений для детектирования объектов на изображении при помощи нейросетей, и, если появится необходимость, создание собственной нейросети и обучение на снимках с инфракрасной камеры.